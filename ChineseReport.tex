\documentclass[a4paper,UTF8]{ctexart}%Chinese
\newcommand{\mytitle}{题目}
    \author{电41班~刘昊天~2014010942}
    \title{\mytitle}
    \renewcommand{\today}{\number\year 年 \number\month 月 \number\day 日}
\usepackage{fontspec}%font
\usepackage{amsmath}%math
\usepackage{amscd}%draw
\usepackage{amssymb}
\usepackage{amsthm}
\usepackage{mathtools}
\usepackage{url}
\usepackage{float}
\usepackage[separate-uncertainty,list-separator={, },list-final-separator={, },list-pair-separator={, },range-phrase={\ensuremath{\sim}}]{siunitx}
\usepackage[all]{xy}%draw
\usepackage{tikz}%draw
\usepackage{booktabs}
\usepackage{graphicx}%insert a pic
\usepackage{subcaption}
\expandafter\def\csname ver@subfig.sty\endcsname{}
\usepackage{multicol}
\usepackage{caption}
    \captionsetup{format=plain,labelsep=period}
\usepackage{ulem}
\usepackage{listings}%code
    \definecolor{mygreen}{rgb}{0,0.6,0}
    \definecolor{mygray}{rgb}{0.5,0.5,0.5}
    \definecolor{mymauve}{rgb}{0.58,0,0.82}
    \lstset{ %
    %   extendedchars=false,%这一条命令可以解决代码跨页时,章节标题,页眉等汉字不显示的问题
        backgroundcolor=\color{white},   % choose the background color
        basicstyle=\footnotesize\ttfamily,        % size of fonts used for the code
        columns=fullflexible,
        breaklines=true,                 % automatic line breaking only at whitespace
        captionpos=b,% sets the caption-position to bottom
        tabsize=4,
        commentstyle=\color{mygreen},    % comment style
        escapeinside={\%*}{*)},          % if you want to add LaTeX within your code
        keywordstyle=\color{blue},       % keyword style
        stringstyle=\color{mymauve}\ttfamily,     % string literal style
        frame=single,
        rulesepcolor=\color{red!20!green!20!blue!20},
        % identifierstyle=\color{red},
        language=matlab
    }
\usepackage{multirow}%table multirow
\usepackage[a4paper,left=2cm,right=2cm,top=2cm,bottom=2cm]{geometry}%paper
\usepackage[colorlinks,linkcolor=blue]{hyperref}
\usepackage{cleveref}
    \crefname{table}{表}{表}
    \Crefname{table}{表}{表}
    \crefname{figure}{图}{图}
    \Crefname{figure}{图}{图}
    \crefname{section}{节}{节}
    \Crefname{section}{节}{节}
    \crefname{equation}{式}{式}
    \Crefname{equation}{式}{式}
    \newcommand{\crefrangeconjunction}{至}
    \newcommand\crefrangepreconjunction{}
    \newcommand\crefrangepostconjunction{}
    \newcommand{\crefpairconjunction}{和}
    \newcommand{\crefmiddleconjunction}{,}
    \newcommand{\creflastconjunction}{和}
    \newcommand{\crefpairgroupconjunction}{和}
    \newcommand{\crefmiddlegroupconjunction}{,}
    \newcommand{\creflastgroupconjunction}{和}
\begin{document}
    \begin{titlepage}
        \begin{flushleft}
            \large \kaishu \mytitle
        \end{flushleft} 
        \begin{center}  
            \kaishu     
            \vfill \Huge
            吉林省水资源利用情况调研及建议\\
            %\Large ——2016年毛泽东思想和中国特色社会主义理论体系概论课程调研\\
            \vfill
            \renewcommand{\ULthickness}{0.1pt}
            \ULdepth=3pt  \LARGE
            \makebox[38mm][s]{提交人姓名} \uline{\hfill 刘昊天 \hfill}\footnote{具体分工见附录\ref{sec:Distribute}。}
            \vspace{1mm}
            \\ \makebox[38mm][s]{提交人信息}\uline{\hfill 2014010942,电41班 \hfill}
            \vspace{1mm}
            \\ \makebox[38mm][s]{其他成员姓名} \uline{\hfill 赵宝旭、杨昊霖、时光 \hfill}
            \vspace{1mm}
            \\ \makebox[38mm][s]{开展地区}\uline{\hfill 吉林省 \hfill}
            \vspace{1mm}
            \\ \makebox[38mm][s]{调研时间}\uline{\hfill 2016 年 2 月 \hfill}
        \end{center}
    \end{titlepage}
    \thispagestyle{empty}
    \tableofcontents
    \newpage
    \pagestyle{headings}
    \setcounter{page}{1}
    \maketitle
    \begin{abstract}
        吉林省属于水资源相对短缺的地区。本调研通过宏观数据统计与微观个体访谈,立体地展现了吉林省水资源的利用情况。本调研从农业、工业、生活三个方面出发,分析了吉林省水资源利用的途径,进而发现其中的经验与问题。文末,本调研从多方面给出了对吉林省水资源利用的建言。
        \\[1\baselineskip] \textbf{关键字:}吉林省,水资源,节水建言
    \end{abstract}

    \newpage
    \bibliography{report}
    \bibliographystyle{unsrt}
\end{document}